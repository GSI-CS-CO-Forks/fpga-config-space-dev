\documentclass[a4paper, 12pt]{article}

\usepackage{draftwatermark}
\usepackage{changepage}
%\usepackage{fullpage}
\usepackage{color}
\usepackage{xcolor}
\usepackage{listings}
\usepackage{array}
\usepackage{multirow}
\usepackage{footnote}
\usepackage{graphicx}

\SetWatermarkScale{4}

\usepackage{caption}
\DeclareCaptionFont{white}{\color{white}}
\DeclareCaptionFormat{listing}{\colorbox{gray}{\parbox{\textwidth}{#1#2#3}}}
\captionsetup[lstlisting]{format=listing,labelfont=white,textfont=white}
\lstset{
	tabsize=4,
	basicstyle=\footnotesize,
	columns=fixed,
}

\parindent0pt
\parskip10pt
\makesavenoteenv{tabular}

\title{Self Descriptive Structures for Logic Cores}
\author{Manohar Vanga (BE/CO/HT), Alessandro Rubini (University of Pavia)}
\date{1 June 2012}
\begin{document}

\maketitle

\tableofcontents
\listoftables
\listoffigures

\pagebreak

\section{Introduction}

This document describes a specification for a series of self descriptive
structures that can be used to provide metadata about logic blocks. This metadata
may be used by higher level abstractions to automatically discover the blocks and
configure them during runtime.

\section{SDWB Structures}

This section describes the important structures that are a part of the SDWB specification.

\subsection{SDWB Table}

The most important structure that needs to be present in a design to support SDWB
based autodiscovery is the SDWB table.

The SDWB table is a contiguous array of records that provides the needed metadata
about devices. There is no restriction placed on where this table is located in
memory and the higher levels only need to know the location of the top-level table
to recursively discover the descriptions of all the blocks.

An SDWB table is composed of multiple SDWB records. There are four different types of
SDWB records that can be used to describe different parts of the design.

\begin{enumerate}
\item Interconnect Record
\item Device Record
\item Bridge Record
\item Integrator Record
\end{enumerate}

\subsection{SDWB Product Info}

The SDWB product info structure is a 40 byte structure that provides information about
the specific product being described. The information provided here is used for device
identification.

\begin{center}
  \begin{savenotes}
    \begin{table}[!ht]\footnotesize
      \caption{SDWB Product Info Structure (40 bytes)}\label{sdwb_product_struct}\centering
        \begin{tabular}{| c | c | l | c | p{5cm} |} \hline
        Offset & Size (bytes) & Name & Value & Description \\ \hline
        0x00 & 8 & VENDOR\_ID & - & 64-bit vendor ID \\ \hline
        0x08 & 4 & DEVICE\_ID & - & 32-bit vendor specific device ID \\ \hline
        0x0C & 4 & VERSION & - & Vendor specific device version number \\ \hline
        0x10 & 4 & DATE & - & The release date (hex format, eg. 0x20120601) \\ \hline
        0x14 & 19 & NAME & - & ASCII device name without NULL terminator \\ \hline
        0x27 & 1 & RECORD\_TYPE & - & Record type byte (see Table \ref{record_type}) \\ \hline
        \end{tabular}
    \end{table}
  \end{savenotes}
\end{center}

\begin{description}
\item[VENDOR\_ID] \hfill \\
This field provides a 64 bit field that identifies the vendor of the device. The vendor may
be an company, organization or an individual. The size of this field has been kept at 64
bits to prevent collisions in the long term.

A registry is still needed to prevent collisions when using community developed designs from
multiple sources and one will be set up in the near future. However, use of the registry is
not mandated. To allow for this, the 64 bit ID space is split into two parts. All ID's with
the most significant bit set are considered free to use. No guarantees are provided regarding
collisions when using this space during integration of designs with externally provided
components.

The second part of the ID space with the most significant bit unset is reserved for use with
the registry.

\item[DEVICE\_ID] \hfill \\
This field specifies a manufacturer defined device ID for the device being described.

\item[VERSION] \hfill \\
This field specifies a manufacturer defined version number for the device. For example, this
may be derived from the information provided by the source code management being used.

\item[DATE] \hfill \\
This field specified the release date of the device being described. This must be a 32-bit hex
format number in the format 0xYYYYMMDD. For example, 0x20120101 specifies the first day of the
year 2012.

\item[NAME] \hfill \\
The ASCII name of the device. No NULL terminator need be provided in this field. The name length
may be a maximum length of 19 characters. Unused bytes should be set to 0x20 (or ASCII SPACE).

\item[RECORD\_TYPE] \hfill \\
Since the product info structure is embedded within various types of record structures, this
field specifies the type of the encapsulating structure. There is a record type for each kind
of SDWB record as well as one to specify an empty record. The various record types are described
in Table \ref{record_type}.

\begin{center}
  \begin{savenotes}
    \begin{table}[!ht]\footnotesize
      \caption{SDWB Record Types}\label{record_type}\centering
        \begin{tabular}{| c | l | p{5cm} |} \hline
        Name & Value & Description \\ \hline
        INTERCONNECT & 0x00 & Interconnect record type \\ \hline
        DEVICE & 0x01 & Device record type \\ \hline
        BRIDGE & 0x02 & Bridge record type \\ \hline
        INTEGRATOR & 0x80 & Integrator record type \\ \hline
        EMPTY & 0xFF & Empty record type \\ \hline
        \end{tabular}
    \end{table}
  \end{savenotes}
\end{center}
\end{description}

\subsection{SDWB Component Info}

The Component Info structure is a 56 byte structure which, in addition to embedding product
information, also additionally provides information regarding the address space used by the
device. When a particular record type requires address space information in addition to product
information, this structure is embedded.

\begin{center}
  \begin{savenotes}
    \begin{table}[!ht]\footnotesize
      \caption{SDWB Component Info Structure (56 bytes)}\label{sdwb_component_struct}\centering
        \begin{tabular}{| c | c | l | c | p{5cm} |} \hline
        Offset & Size (bytes) & Name & Value & Description \\ \hline
        0x00 & 8 & ADDR\_BEGIN & - & The start address of the component \\ \hline
        0x08 & 8 & ADDR\_LAST & - & The last valid address of the component \\ \hline
        0x10 & 40 & PRODUCT & - & SDWB Product Info structure (see Table \ref{sdwb_product_struct} \\ \hline
        \end{tabular}
    \end{table}
  \end{savenotes}
\end{center}

\begin{description}
\item[ADDR\_BEGIN] \hfill \\
The start address of the memory area occupied by the device.

\item[ADDR\_LAST] \hfill \\
The last valid address in memory that is occupied by the device. For example, a device that takes
up the entire 64-bit space has its last address at 0xffffffffffffffff

\item[PRODUCT] \hfill \\
This is the embedded 40 byte product info structure as described in Table \ref{sdwb_product_struct}.
\end{description}

\subsection{SDWB Records}

This subsection describes the different SDWB records that compose the SDWB table structure. These
structures must be instantiated by designers for each logic block in their design and compiled into
a contiguous SDWB table that gets placed in the bus memory.

\subsubsection{Interconnect Record}

The interconnect record describes the high level bus that has child devices connected to it. This
structure should be the "header" or first record in the SDWB table. It provides, among other things,
information regarding the SDWB table (eg. the number of following records).

\begin{center}
  \begin{savenotes}
    \begin{table}[!ht]\footnotesize
      \caption{SDWB Interconnect Record Structure}\label{sdwb_interconnect_struct}\centering
        \begin{tabular}{| c | c | l | c | p{5cm} |} \hline
        Offset & Size (bytes) & Name & Value & Description \\ \hline
        0x00 & 4 & MAGIC & 0x53445742 & Magic value for checking validity \\ \hline
        0x04 & 2 & NUM\_RECORDS & - & Number of records in this SDWB table \\ \hline
        0x06 & 1 & SDWB\_VERSION & 0x1 & SDWB format version. Current is 0x1 \\ \hline
        0x07 & 1 & SDWB\_BUS\_TYPE & - & The last valid address of the component \\ \hline
        0x08 & 56 & COMPONENT & - & SDWB Component Info structure (see Table \ref{sdwb_component_struct} \\ \hline
        \end{tabular}
    \end{table}
  \end{savenotes}
\end{center}

\begin{description}
\item[MAGIC] \hfill \\
This field is a unique number to ensure that the record being read is valid. The default
value for this in this version of the specification is 0x53445742 or the ASCII characters
"SDWB".

\item[NUM\_RECORDS] \hfill \\
This field specifies the number of records in the table. The number must take into account
this header (interconnect) record as well. The other records following the header may be
any combination of device, bridge and integrator records.

\item[SDWB\_VERSION] \hfill \\
This is the record format version. In the current version of the specification this is the
value 0x1.

\item[SDWB\_BUS\_TYPE] \hfill \\
This field specifies the bus type. This field is used when decoding the bus specific information
inside a device record (see below). In the current version of the specification, the only
supported bus is Wishbone. Other buses may be added in the future.

\begin{center}
  \begin{savenotes}
    \begin{table}[!ht]\footnotesize
      \caption{SDWB Bus Types}\label{bus_type}\centering
        \begin{tabular}{| c | l | p{5cm} |} \hline
        Name & Value & Description \\ \hline
        WISHBONE & 0x00 & Specifies a Wishbone bus type \\ \hline
        \end{tabular}
    \end{table}
  \end{savenotes}
\end{center}

\item[COMPONENT] \hfill \\
An interconnect record has a component info structure embedded within it which can be used
to provide additional meta-data regarding the interconnect itself.
\end{description}

\subsubsection{Integrator Record}

An integrator record gives meta-data about the aggregate product of the bus. For example, consider
a manufacturer that takes components from various vendors and combines them with a standard bus
interconnect. This aggregate product can be described by an SDWB integrator record.

This record is important as it helps to clearly distinguish product information of devices and of
their aggregated products. Since this is the only main requirement of the integrator record, it
only contains a product info structure without the additional address space information (as this
can easily be gleaned from the records for the specific devices that compose the aggregate). The
structure of the integrator record is described in Table \ref{sdwb_integrator_struct}.

\begin{center}
  \begin{savenotes}
    \begin{table}[!ht]\footnotesize
      \caption{SDWB Integrator Record Structure}\label{sdwb_integrator_struct}\centering
        \begin{tabular}{| c | c | l | c | p{5cm} |} \hline
        Offset & Size (bytes) & Name & Value & Description \\ \hline
        0x00 & 24 & RESERVED & - & Reserved bytes \\ \hline
        0x18 & 40 & PRODUCT & - & SDWB Product Info structure (see Table \ref{sdwb_product_struct} \\ \hline
        \end{tabular}
    \end{table}
  \end{savenotes}
\end{center}

\begin{description}
\item[PRODUCT] \hfill \\
An integrator record has a product info structure embedded within it which can be used
to provide meta-data that aids in identifying the aggregating interconnect.
\end{description}

\subsubsection{Device Record}

This record type describes a single device or logic block mapped into the memory of the
bus. In a correct implementation, one device record should exist for each device that is
connected to the bus. The structure of the device record structure is shown below in Table
\ref{sdwb_dev_struct}.

\begin{center}
  \begin{savenotes}
    \begin{table}[!ht]\footnotesize
      \caption{SDWB Device Record Structure}\label{sdwb_dev_struct}\centering
        \begin{tabular}{| c | c | l | c | p{5cm} |} \hline
        Offset & Size (bytes) & Name & Value & Description \\ \hline
        0x00 & 2 & ABI\_CLASS & - & The ABI class of the device (0 = Custom Device) \\ \hline
        0x02 & 1 & ABI\_VER\_MAJOR & - & The ABI major version \\ \hline
        0x03 & 1 & ABI\_VER\_MINOR & - & The ABI minor version \\ \hline
        0x04 & 4 & BUS\_SPECIFIC & - & Bus specific field \\ \hline
        0x08 & 56 & COMPONENT & - & SDWB Component Info structure (see Table \ref{sdwb_component_struct} \\ \hline
        \end{tabular}
    \end{table}
  \end{savenotes}
\end{center}

\begin{description}
\item[ABI\_CLASS] \hfill \\
The ABI class, if specified, tells the kind of standard interface that the device provides. The
currently available types of ABI classes are specified in Table \ref{abi_class_list}.

\begin{center}
  \begin{savenotes}
    \begin{table}[!ht]\footnotesize
      \caption{ABI Classes}\label{abi_class_list}\centering
        \begin{tabular}{| c | l | p{5cm} |} \hline
        Value & ABI Class & Description \\ \hline
        0 & Custom Device & Used when the device does not conform to any standard interface \\ \hline
        \end{tabular}
    \end{table}
  \end{savenotes}
\end{center}

\item[ABI\_VER\_MAJOR] \hfill \\
This is the major version number of the ABI class. Standard interfaces are not compatible between
major version changes.

\item[ABI\_VER\_MINOR] \hfill \\
This is the minor version number of the ABI class. Standard interfaces are compatible between
minor version changes.

\item[BUS\_SPECIFIC] \hfill \\
This is a 4-byte field that holds bus-specific information. The bus specific information for different
supported buses is described in Table \ref{bus_specific}.

\begin{center}
  \begin{savenotes}
    \begin{table}[!ht]\footnotesize
      \caption{Wishbone -- Bus Specific Information}\label{bus_specific}\centering
        \begin{tabular}{| c | l | p{5cm} |} \hline
        Bit Offset & Name & Description \\ \hline
        0 & ENDIAN & Specifies the endianness of the device (0 = big endian, 1 = little endian) \\ \hline
        \end{tabular}
    \end{table}
  \end{savenotes}
\end{center}

\item[COMPONENT] \hfill \\
An embedded component info structure for providing meta-data about the device being described.
\end{description}

\subsubsection{Bridge Record}

A bridge record is used to describe a nested bus within the same address space. This is different from
a bus controller which provides access to an entirely different address space altogether. Bus
controllers should be treated and handled as devices and not bridges. The structure of the bridge record
is shown in Table \ref{sdwb_bridge_struct}.

\begin{center}
  \begin{savenotes}
    \begin{table}[!ht]\footnotesize
      \caption{SDWB Bridge Structure}\label{sdwb_bridge_struct}\centering
        \begin{tabular}{| c | c | l | c | p{5cm} |} \hline
        Offset & Size (bytes) & Name & Value & Description \\ \hline
        0x00 & 8 & SDWB\_CHILD & - & The relative address of the nested SDWB table \\ \hline
        0x08 & 56 & COMPONENT & - & SDWB Component Info structure (see Table \ref{sdwb_component_struct} \\ \hline
        \end{tabular}
    \end{table}
  \end{savenotes}
\end{center}

\begin{description}
\item[SDWB\_CHILD] \hfill \\
This field gives the location of the nested bus' SDWB table. This address is a relative address
with respect to the start of the nested bus' address space (stored in the component info structure).

\item[COMPONENT] \hfill \\
An embedded component info structure for providing meta-data about the bridge being described.
\end{description}

\end{document}
